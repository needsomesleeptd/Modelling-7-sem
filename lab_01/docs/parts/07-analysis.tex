\chapter{Теоретическая часть}


\section{Условие лабораторной}

Разработать программу для построения графиков функции распределения и функции плотности распределения для следующих распределений: 
\begin{itemize}
	\item равномерное распределение;
	\item пуассоновское распределение (вариант 1).
\end{itemize} 

\section{Равномерное распределение}

Случайная величина $X$ имеет \textit{равномерное распределение} на отрезке~$[a,~b]$, если ее плотность распределения~$f(x)$ равна:
\begin{equation}
	p(x) =
	\begin{cases}
		\displaystyle\frac{1}{b - a}, & \quad \text{если } a \leq x \leq b;\\
		0,  & \quad \text{иначе}.
	\end{cases}
\end{equation}

При этом функция распределения~$F(x)$ равна:

\begin{equation}
	F(x) =
	\begin{cases}
		0,  & \quad x < a;\\
		\displaystyle\frac{x - a}{b - a}, & \quad a \leq x \leq b;\\
		1,  & \quad x > b.
	\end{cases}
\end{equation}

Обозначение: $X \sim R[a, b]$.

\clearpage

\section{Пуассоновское распределение}

Дискретная случайная величина $X$, возможными значениями которой являются $X=m$, где $(m=0,1,2...)$, а вероятности соответствующих значений определяются по формуле Пуассона
\begin{equation}
	P(x = m) = \frac{\lambda^{(m)} e^{(-\lambda)}}{m!}	
\end{equation}
называется пуассоновской случайной величиной с параметром $\lambda$.

При этом функция распределения~$F(k)$ равна:

\begin{equation}
	F(k; \lambda) = \frac{\Gamma(k+1, \lambda)}{k!}
\end{equation}



Обозначение: $X \sim N(m, \sigma^2)$.

