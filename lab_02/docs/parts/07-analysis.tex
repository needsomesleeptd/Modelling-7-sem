\chapter{Условие лабораторной}

Написать программу, которая позволяет определить время пребывания сложной системы в каждом из состояний в установившемся режиме работы. Количество состояний $\le 10$.

Реализовать интерфейс, который позволяет указать количество состояний и значения матрицы вероятностей переходов, а также отображает результаты работы программы: время стабилизации вероятности каждого состояния и стабилизировавшееся значение вероятности каждого состояния.

\chapter{Теоретическая часть}

Случайный процесс называется марковским случайным прооцессом, если для каждомго момента времени вероятность люого состояния системы в будущем зависит только от состояния системы в настоящем и не завписит от того, когда и каким образом система пришла в это состояние.

Для марковского случайного процесса составляют уравнения Колмогорова, следуя следующему правилу: в левой части каждого уравнения находится производная функции, отражающей вероятность нахождения системы в $i$-ом состоянии, в правой части находится столько членов, сколько трелок связано с данным состоянием в направленном графе состояний, причём если стрелка выходит из состояния, член имеет знак минус, если в состояние, знак плюс. 

Таким образом, в правой части находится сумма произведений вероятностей всех состояний, переводящих систему в данное состояние, на интенсивности соответствующих переходов, минус суммарная интенсивность всех переходов, выводящих систему из данного состояния, умноженная на вероятность данного состояния. Уравнение Колмогорова для состояния с номером $i$ будет иметь следующий вид:

\begin{equation}
	p^{'}_{i}(t) = \sum_{j=1}^{n}\lambda_{ji}p_{j}(t) - p_{i}(t)\cdot \sum_{j=1}^{n}\lambda_{ij},
\end{equation}

где:

$n$ --- число состояний рассматриваемой ситемы;

$\lambda_{ij}$ --- интенсивность перехода системы из $i$-го состояния в $j$-ое.

Предельная вероятность состояния --- среднее относительное время нахождения системы в данном состоянии.
Для определения предельных вероятностей необходимо решить систему уравнений Колмогорова. 
Поскольку по условию задачи рассматриваемый марковский процесс является стационарным, производные вероятностей заменяются нулями. 
При этом одно из уравнений в системе необходимо заменить уравнением нормировки: $\sum_{i = 1}^{n}p_{i}(t) = 1$, где $n$ --- количество состояний системы.