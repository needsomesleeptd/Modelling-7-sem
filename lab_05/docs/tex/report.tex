\documentclass{bmstu}

\begin{document}



\makereporttitle
{Информатика и системы управления (ИУ)}
{Программное обеспечение ЭВМ и информационные технологии (ИУ7)}
{лабораторной работе №~5}
{Моделирование}
{Моделирование работы информационного центра}
{}
{Разин~А.В./ИУ7-74Б}
{Рудаков~И.В.}

\setcounter{page}{2}
\renewcommand{\contentsname}{Содержание} 
\tableofcontents

\chapter{Задание}

В информационный центр приходят клиенты через интервалы времени 10±2 минуты. Если все три имеющихся оператора заняты, клиенту отказывают в обслуживании. Операторы имеют разную производительность и могут обеспечивать обслуживание среднее запросы за 20±5, 40±10, 40±20 минут. Клиенты стремятся занять свободного оператора с максимальной производительностью. Полученные запросы сдаются в приемные накопители, откуда они выбираются для обработки. На первый компьютер -- запросы от первого и второго операторов, на второй компьютер -- от третьего оператора. Время обработки на первом и втором компьютере равны соответственно 15 и 30 минутам. Смоделировать процесс обработки 300 запросов. Определить вероятность отказа.

\chapter{Теоретическая часть}

\section{Схемы модели}

На рисунке \ref{img:blockDiagram} представлена структурная схема модели.


\includeimage
{blockDiagram} % Имя файла без расширения (файл должен быть расположен в директории inc/img/)
{f} % Обтекание (без обтекания)
{H} % Положение рисунка (см. figure из пакета float)
{0.7\textwidth} % Ширина рисунка
{Структурная схема модели} % Подпись рисунка


В процессе взаимодействия клиентов с информационным центром возможно два режима работы:

\begin{itemize}
    \item режим нормального обслуживания, когда клиент выбирает одного из свободных операторов, отдавая предпочтение тому, у кого максимальная производительность;
    \item режим отказа клиенту в обслуживании, когда все операторы заняты.
\end{itemize}

На рисунке \ref{img:queuingSystems} представлена схема модели в терминах систем массового обслуживания (СМО).

\includeimage
{queuingSystems} % Имя файла без расширения (файл должен быть расположен в директории inc/img/)
{f} % Обтекание (без обтекания)
{H} % Положение рисунка (см. figure из пакета float)
{0.7\textwidth} % Ширина рисунка
{Схема модели в терминах СМО} % Подпись рисунка


\section{Переменные и уравнение имитационной модели}

\textbf{Эндогенные переменные:}

\begin{itemize}
    \item время обработки задания $i$-ым оператором;
    \item время решения задания на $j$-ом компьютере.
\end{itemize}

\textbf{Экзогенные переменные:}

\begin{itemize}
    \item $n0$ — число обслуженных клиентов;
    \item $n1$ — число клиентов, получивших отказ.
\end{itemize}

Вероятность отказа в обслуживании клиента будет вычисляться как:

\begin{equation}
    P = \frac{n_0}{n_0 + n_1}
\end{equation}

\chapter{Результаты работы}


\section{Демонстрация работы программы}

На рисунке \ref{img:result} представлен пример работы программы.

\includeimage
{result} % Имя файла без расширения (файл должен быть расположен в директории inc/img/)
{f} % Обтекание (без обтекания)
{H} % Положение рисунка (см. figure из пакета float)
{1\textwidth} % Ширина рисунка
{Результат работы программы} % Подпись рисунка


\end{document}
