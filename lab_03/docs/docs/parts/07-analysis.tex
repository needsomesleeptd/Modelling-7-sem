\chapter{Условие лабораторной}

Необходимо взять одно-, двух- и трехразрядные числа, сгенерированные табличным и алгоритмическим способами (три столбца). Дать возможность
ввести 10 любых чисел и затем под каждым из столбцов вывести число, показывающее случайность данной последовательности — разработать количественный критерий случайности для чисел, сгенерированных табличным и алгоритмическим способами. Если числа будут подчиняться какому-либо закону, то они уже не случайны.

\chapter{Теоретическая часть}

\section{Методы получения последовательности случайных чисел}

Существует три метода получения последовательности случаных чисел:
\begin{enumerate}
	\item аппаратный;
	\item табличный;
	\item алгоритмический. 
\end{enumerate}

\section{Табличный способ}

Табличный способ подразумевает использование файла (таблицы), содержащего случайные числа.

\section{Алгоритмиический способ}

В качестве алгоритмического способа генерации псевдослучайных чисел был выбран способ генерации при помощи генератора равномерных вихревых последовательностей целых случайных величин без запоминающего массива.
Данный способ описан Алексеем Фёдоровичем Деоном в статье ''Генератор равномерных вихревых последовательностей случайных величин без запоминающего массива'', а также в статье ''Вихревой генератор случайных величин Пуассона  по технологии кумулятивных частот'', изданных в журнале ''Вестник приборостроения'' в 2020 году в МГТУ им. Н. Э. Баумана.


\section{Критерий случайности}

Был составлен следующий критерий случайности последовательности: рассчитывалась разность разностей ближайших значений  последовательности $S$ по формуле \ref{for:data}, после чего вычислялась матожидание и среднеквадратичное отклонение полученных случайных величин ($H(S)_b$), отношение среднеквадратичного отклонения и матожидания является метрикой~\ref{for:koeff}.

\begin{equation}
	\label{for:data}
	H(S)_{b} = \sum_{i=1}^{n-1} ||S_{i} - S_{i - 1}| - |S_{i+1} - S_{i}||
\end{equation}

где $n$ --- количество встречающихся в последовательности чисел, $S_{i}$~---~значение элемента на позиции~$i$. 

\begin{equation}
	\label{for:koeff}
	r = \frac{\sigma(H)}{M(H)} 
\end{equation}




Чем ближе к нулю находится значение коэффициента $r$, тем менее случайные значения последовательности $S$.

