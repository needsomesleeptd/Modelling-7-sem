\documentclass{bmstu}

\begin{document}

\makereporttitle
{Информатика и системы управления (ИУ)}
{Программное обеспечение ЭВМ и информационные технологии (ИУ7)}
{лабораторной работе №~4}
{Моделирование}
{Моделирование работы системы массового обслуживания}
{}
{Разин~А.В./ИУ7-74Б}
{И.В. Рудаков}


\setcounter{page}{2}
\renewcommand{\contentsname}{Содержание} 
\tableofcontents

\chapter{Задание}

Промоделировать систему, состоящую из генератора, памяти и обслуживающего аппарата. Генератор подает сообщения, распределенные по равномерному закону, они приходят в память и выбираются на обработку по закону из ЛР1 (Пуассона). Количество заявок конечно и задано. Предусмотреть случай, когда обработанная заявка возвращается обратно в очередь. Определить оптимальную длину очереди, при которой не будет потерянных сообщений. Реализовать двумя способами: используя пошаговый и событийный подходы.



\chapter{Теоретическая часть}

\section{Равномерное распределение}

Функция равномерного распределения:

\begin{equation}
    F(x) =
    \begin{cases}
            0, x < a, \\
            \begin{aligned}
                \frac{x -  a}{b - a}, x \in [a, b], 
            \end{aligned}\\
            0, x > b. \\
    \end{cases}
\end{equation}

Функция плотности равномерного распределения:

\begin{equation}
    f(x) =
    \begin{cases}
            \begin{aligned}
                \frac{1}{b - a}, x \in [a, b], 
            \end{aligned}\\
            0, else. \\
    \end{cases}
\end{equation}

\section{Распределение Пуассона}

Дискретная случайная величина $X$, возможными значениями которой являются $X=m$, где $(m=0,1,2...)$, а вероятности соответствующих значений определяются по формуле Пуассона
\begin{equation}
	P(x = m) = \frac{\lambda^{(m)} e^{(-\lambda)}}{m!}	
\end{equation}
называется пуассоновской случайной величиной с параметром $\lambda$.

При этом функция распределения~$F(k)$ равна:

\begin{equation}
	F(k; \lambda) = \frac{\Gamma(k+1, \lambda)}{k!}
\end{equation}


В данных формулах $\lambda$ и $k$ --- положительные параметры распределения $(\lambda \geqslant 0; k = 1, 2, ...)$;
$x \geqslant 0$.

\clearpage

\section{Принципы управляющей программы}

\subsection{Пошаговый подход}

Заключается в последовательном анализе состояний всех блоков системы в момент $t$ + $\Delta t$ по заданному состоянию в момент $t$. При этом новое состояние блоков определяется в соответствии с их алгоритмическим описанием с учетом действующих случайных факторов. В результате этого анализа принимается решение о том, какие системные события должны имитироваться на данный момент времени. Основной недостаток: значительные затраты машинных ресурсов, а при недостаточном малых $\Delta t$ появляется опасность пропуска события.

\subsection{Событийный принцип}

Характерное свойство модели системы обработки информации: состояние отдельных устройств изменяется в дискретные моменты времени, совпадающие с моментами поступления сообщения, окончания решения задачи, возникновения аварийных сигналов и т. д. При использовании событийного принципа состояния всех боков системы анализируется лишь в момент появления какого-либо события. Момент наступления следующего события определяется минимальным значением из списка будущих событий, представляющий собой совокупность моментов ближайшего изменения состояния каждого из блоков. Момент наступления следующего события определяется минимальным значением из списка событий.

\chapter{Результаты работы}



\section{Демонстрация работы программы}

На рисунках \ref{img:P_0} - \ref{img:P_100} представлены примеры работы программы.

\includeimage
{P_0} % Имя файла без расширения (файл должен быть расположен в директории inc/img/)
{f} % Обтекание (без обтекания)
{H} % Положение рисунка (см. figure из пакета float)
{0.5\textwidth} % Ширина рисунка
{Пример работы без обратной связи} % Подпись рисунка




\includeimage
{P_5} % Имя файла без расширения (файл должен быть расположен в директории inc/img/)
{f} % Обтекание (без обтекания)
{H} % Положение рисунка (см. figure из пакета float)
{0.5\textwidth} % Ширина рисунка
{Пример работы с вероятностью возврата заявки 0.5} % Подпись рисунка



\includeimage
{P_100} % Имя файла без расширения (файл должен быть расположен в директории inc/img/)
{f} % Обтекание (без обтекания)
{H} % Положение рисунка (см. figure из пакета float)
{0.5\textwidth} % Ширина рисунка
{Пример работы с вероятностью возврата заявки 1} % Подпись рисунка
\end{document}
