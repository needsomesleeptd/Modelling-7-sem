\chapter{Условие лабораторной}

Смоделировать работу системы массового обслуживания на языке имитационного моделирования GPSS. Определить минимальный размер очереди,
при котором не будет потерянных заявок.
Генератор создаёт заявки, распределённые по равномерному закону.
Обслуживающий аппарат обрабатывает заявки по Пуассоновскому закону (вариант из
лабораторной работы №1). Предусмотреть возможность возврата обработанной
заявки в очередь.


\chapter{Теоретическая часть}

\section{Используемые законы распределения}

\subsection*{Закон появления сообщений}

Согласно заданию лабораторной работы для генерации сообщений используется равномерный закон распределения.
Случайная величина $X$ имеет \textit{равномерное распределение} на отрезке~$[a,~b]$, если ее плотность распределения~$f(x)$ равна:
\begin{equation}
	p(x) =
	\begin{cases}
		\displaystyle\frac{1}{b - a}, & \quad \text{если } a \leq x \leq b;\\
		0,  & \quad \text{иначе}.
	\end{cases}
\end{equation}

При этом функция распределения~$F(x)$ равна:

\begin{equation}
	F(x) =
	\begin{cases}
		0,  & \quad x < a;\\
		\displaystyle\frac{x - a}{b - a}, & \quad a \leq x \leq b;\\
		1,  & \quad x > b.
	\end{cases}
\end{equation}

Обозначение: $X \sim R[a, b]$.

Интервал времени между появлением $i$-ого и $(i - 1)$-ого сообщения по равномерному закону распределения вычисляется следующим образом:

\begin{equation}
	T_{i} = a + (b - a) \cdot R,
\end{equation}

\noindent где $R$ --- псевдослучайное число от 0 до 1.

\subsection*{Закон обработки сообщений}
Сообщения обрабатываются в соответствии с законом распределения Пуассона.


Дискретная случайная величина $X$, возможными значениями которой являются $X=m$, где $(m=0,1,2...)$, а вероятности соответствующих значений определяются по формуле Пуассона
\begin{equation}
	P(x = m) = \frac{\lambda^{(m)} e^{(-\lambda)}}{m!}	
\end{equation}
называется пуассоновской случайной величиной с параметром $\lambda$.

При этом функция распределения~$F(k)$ равна:

\begin{equation}
	F(k; \lambda) = \frac{\Gamma(k+1, \lambda)}{k!}
\end{equation}

\section{GPSS}

Язык GPSS – общецелевая система моделирования.

Транзакты представляют собой описание динамических процессов в
реальных системах. Они могут описывать как реальные физические объекты,
так и нефизические, например, канальная программа. Транзакты можно
генерировать и уничтожать в процессе моделирования. Основным атрибутом
любого транзакта является число параметров (от 0 до 1020).

Динамическими объектами являются транзакты, которые представляют собой единицы исследуемых потоков и производят ряд определённых
действий, продвигаясь по фиксированной структуре, представляющей собой
совокупность объектов других категорий.

Операционный объект. Блоки задают логику функционирования
системы и определяют маршрут движения транзактов между объектами
аппаратной категории. Это абстрактные элементы, на которые может быть
декомпозирована структура реальной системы. Воздействуя на эти объекты,
транзакты могут изменять их состояния и оказывать влияние на движение
других объектов.

Вычислительный объект. Служит для описания таких операций
в процессе моделирования, когда связи между элементами моделируемой
системы наиболее просто выражаются в виде математических соотношений.

К статическим объектам относятся очереди и таблицы, служащие
для оценок влияющих характеристик.

Рассмотрим некоторые команды:
\begin{enumerate}
	\item \textbf{GENERATE} --- команда, вводящая транзакты в модель.
	\item \textbf{TERMINATE} ---  команда, удаляющая транзакт.
	\item \textbf{QUEUE} --- команда, помещающая транзакт в конец очереди.
	\item \textbf{DEPART} --- команда, удаляющая транзакт из очереди.
	\item \textbf{SEIZE} --- команда, занимающая канал обслуживания.
	\item \textbf{RELEASE} --- команда, освобождающая канал обслуживания.
	\item \textbf{ADVANCE} ---  команда, задерживающая транзакт.
	\item \textbf{TRANSFER} --- команда, изменяющая движение транзакта в модели.
	\item \textbf{START} --- команда, управляющая процессом моделирования.
\end{enumerate}